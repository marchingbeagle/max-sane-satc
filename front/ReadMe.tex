faça um resumo completo sobre as informações do trabalho que fizemos, 
Foi usado front com typescript algumas coisas são diferentes do padrão usado no react mas funciona da mesma forma e a logica pra react é a mesma, para iniciar o projeto tem que entrar na pasta de Front atraves do terminal com o comando CD Front a paginação não ta complicada e bem simples de saber onde é oq, codigos com nome Page.tsx é onde será exibido as informações de tela sendo componentes, ou proprio codigos, seria a parte do 'use client' do next, normalmente os erros que podem acontecer são referentes a o tipo de use que ta sendo puxado, como useState não tem como utilizar no lado do servidor, useEffect tb n tem, pq são ações referente a interação do usuario. a biblioteca que estou utilizando para front é a shardUI, documentação dela é bem simples não tem oq ter duvida mesma coisa de bootstrap, apenas facilitei a padronização do site com um framework de css fora o tailwind, o mesmo tambem ja deixa por si só muitas telas responsivas. os comando principais são de crud no sistema, bom n temos muito oque questionar sobre isso até pq o sistema em si não vai ser por assim dizer usado, a gente ta passando esse trampo meio atoa então é mais a parte do index. 
A pagina principal n tem muito uma puta estrutura feita é mais tudo visual o botão de whatsapp foi feito a mão e um SVG por cima então né, o padrão ta feito tudo la no Figma não foge daquilo, 

Usando os blocos de codigo dos ultimos meses fale sobre o projeto e explique melhor mais informações, melhore o texto e adicione informações relevantes tanto para a usabilidade quanto o entendimento para o time
Tela de Login: Permite que os usuários façam login na aplicação, autenticando suas credenciais para acessar as funcionalidades do sistema.
Utilizando NextAuth padroniza as informções fazendo com que pegue de maneira mais segura e padronizada, oque um form de Login precisaria, alem de auxiliar com processos que utilizamos para a validação de login, commo devolver um Token Jwt e validar credenciais automaticamente

Dashboard: Apresenta um resumo das informações essenciais e estatísticas relevantes do sistema, fornecendo uma visão geral do desempenho e atividades recentes.

Tela de CRUD de Produtos: Permite aos usuários adicionar, visualizar, editar e excluir produtos do sistema, gerenciando assim o catálogo de produtos da empresa.
Utilizando Json-server por base então tem que utilizar o comando npm run json-server

Tela de CRUD de Pedidos: Permite aos usuários criar, visualizar, editar e excluir pedidos de clientes, facilitando a gestão e acompanhamento das vendas.

Tela de Relatórios: Apresenta relatórios detalhados sobre vendas, estoque, e outras métricas relevantes, fornecendo insights valiosos para a tomada de decisões.

Tela de Configurações: Permite aos usuários configurar preferências pessoais e ajustar configurações do sistema conforme necessário.